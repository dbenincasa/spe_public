\documentclass[11pt]{amsart}
\usepackage{geometry}                % See geometry.pdf to learn the layout options. There are lots.
\geometry{letterpaper}                   % ... or a4paper or a5paper or ... 
%\geometry{landscape}                % Activate for for rotated page geometry
%\usepackage[parfill]{parskip}    % Activate to begin paragraphs with an empty line rather than an indent
\usepackage{graphicx}
\usepackage{amssymb}
\usepackage{epstopdf}
\usepackage{hyperref}
\DeclareGraphicsRule{.tif}{png}{.png}{`convert #1 `dirname #1`/`basename #1 .tif`.png}

\title{Brief Article}
\author{The Author}
%\date{}                                           % Activate to display a given date or no date

\begin{document}
\maketitle
%\section{}
%\subsection{}

\section{Introduction}
We write a brief paper meant to support \href{http://www.mhpc.it/}{MHPC} 
lectures on scientific programming environment. The goal of this classes 
is to provide students scientific programming tools to develop scientific applications. 
The author is an engineer, with a Ph.D. in numerical analysis and hopefully these brief 
notes will hold the best habits of these two education paths. Achieving my goal no 
matter what, and in the most efficient way. Being passionate about sophisticated 
mathematical structures. Finite Elements is method that allows me to express my passion 
about mathematics, and meanwhile teach best programming practices. I apologise 
I can't teach programming something that does not stimulate my curiosity.

\section{Strong Problem}
Scientists solve problems. Among the pletora of existing problems 
we need to pick one, simple enough to be solved during te time dedicated to a single 
class, and sophisticated enough to be representative for more sophisticated ones. 
The answer is the Poisson's problem,
modeling the diffusion of temperature in a body.

Given a domain $\Omega\subset \mathbb{R}^2$.
Find $u$ such that:
\[
\left\{
\begin{array}{ll}
u -\Delta u = f & \mathrm{in}\ \Omega \\
\nabla u \cdot \mathbf{n} = 0  & \mathrm{on}\ \partial\Omega
\end{array}
\right.
\]
Just a couple of remarks:
\[
\nabla u \cdot \mathbf{n}  = \partial_{x} u \cdot n_x+ \partial_{y} u \cdot n_y,
\]
\[
\Delta u = \partial_{xx} u + \partial_{yy} u,
\]
where $x$ and $y$ are the two coordinates in $\mathbb{R}^2$, and $\mathbf{n}$ 
is the outer normal to the domain.

In other words, the Poisson's problem is asking for a type of solution 
that is characterised by having two continuous derivatives, in mathematical 
sense we ask for $u\in C^2$. \emph{It can be proved} that such a solution exists and 
is unique. It is also true that such a solution can be \emph{found} in simple and
specific cases. In the next section we explore an alternative strategy.

\section{Continous Weak Problem}
In the previous section we looked at the Poisson's problem and noticed that searching 
for its solution in a strong form requires looking for a $C^2$ solution. This requirement 
is that strong, that can be fulfilled only in simple cases. 
The Finite Element (Method) si not really a method, it is rather the art of looking 
for solutions in vector spaces simpler than the one needed by the strong form.

We understand that a key role is played by the space $V$ substituting 
the $C^2$ space. Assuming $V$ is continuous (has an infinite set of basis 
functions) the first idea is to project the problem onto this particular space. 
Mathematically this is written as:
\[
\int_\Omega u\, v -\int_\Omega \Delta u\, v = \int_\Omega f\, v \quad \forall v \in V
\]
If the projection procedure in functional spaces looks cryptical 
we can compare it to the geometrical projection. Consider $u$ a vector 
and $V$ the $\mathbb{R}^2$ space. The projection of $u$ onto $V$ is
\[
u = u_1 \cdot \mathbf{e}_1 + u_2 \cdot \mathbf{e}_2,
\]
being $\mathbf{e}_1$, $\mathbf{e}_2$ the basis vectors for $V = \mathbb{R}^2$.
The analogy here should be clear once we think that $\mathbf{e}_i$ play the 
same role as $v$.  

If we want to really get rid of the second derivatives in our problem we can 
integrate by parts:
\[
\int_\Omega u\, v +
\int_\Omega \nabla u \cdot \nabla v - 
\int_{\partial \Omega} u \,\nabla v \cdot \mathbf{n} = \int_\Omega f\, v.
\]
The boundary integral term looks more nasty than it actually is. 
Nevertheless we have picked a problem simple enough to postpone the 
discussion on boundary conditions. On the boundary integral we have just
the same flux term that we asked to be zero at the boundary, meaning, our 
integral goes to zero.
\[
\int_\Omega u\, v +
\int_\Omega \nabla u \cdot \nabla v = \int_\Omega f\, v.
\]

For simplicity, we look for $u$ in the very same space as $v$. 
Notice that this is a choice that works pretty well in this case, but
remember, Finite Elements is an art, not a method. The weak formulation 
of our original problem, looks like:

Find $u\in V$, such that:
\[
\int_\Omega u\, v +
\int_\Omega \nabla u \cdot \nabla v = \int_\Omega f\, v.
\]

The relation between the weak problem and the strong one 
is clarified by the Lax-Milgram theorem. This proofs 
that the weak form is equivalent to the strong one, meaning 
we can approximate the continuous weak problem to get the solution 
of the strong one.


\end{document}  